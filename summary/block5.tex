\subsection{Basic Terminology}
\begin{description}
	\item[Release Time $r_i$ (aperiodic tasks)]
	\item[Computation Time $C_i$]
	\item[Deadline $d_i$]
	\item[Slack time $X_i$] = $d_i - r_i - C_i$ max time that can pass until
	this task \textbf{must} be scheduled.
	\item[Laxity] deadline - remaining computation time
	\item[Start Time $s_i$]
	\item[Finishing Time $f_i$]
	\item[Lateness $L_i$] = $f_i - d_i$
	\item[Exceeding Time $E_i$] = $\max(0, L_i)$
	\item[Phase $\Phi_i$] time before the periodic task arrives for the first
	time.
	\item[Period $T_i$] time between consecutive arrivals of a periodic task.
	\item[Relative Deadline $D_i$] Time from release time to respective deadline.
	\item[Release Time (periodic tasks)] In $k$th period: $\Phi_i + k T_i$
	(starting with 0th period).
	\item[Synchronous] $\forall i: \ r_i = 0$
	\item[Independent] No predecessor relation; task can be scheduled in any
	order.
	\item[Shared Resource] Can be used by multiple processes (at once)
	\item[Mutually Exclusive Resource]Can be used by multiple processes one
	after each other.
	\item[Critical Section] Mutex code.
	\item[Semaphore] Operations \texttt{wait} and \texttt{signal}
	\item[Preemptive] \textit{Can} be interrupted.
	\item[Feasible Schedule] Can be executed without violating specified
	properties (\eg not missing deadlines).
\end{description}

\subsection{Cost Functions}
\begin{description}
	\item[Avg Response Time] \ 
	\[ \bar{t_r} = \frac{1}{n}\sum_{i=1}^n(f_i-r_i) \]
	\item[Total Completion Time] \ 
	\[ t_c = \max_i(f_i) - \min_i(r_i) \]
	\item[Weighted Sum of Response Times]\ 
	\[ t_w = \frac{1}{n}\sum_{i=1}^nw_i(f_i-r_i) \]
	\item[Maximum Lateness] \ 
	\[ L_{\max} = \max_i(f_i - d_i) \]
	\item[Number of Late Tasks] \ 
	\[ N_{\text{late}} = \sum_{i=1}^n \INDICATOR{f_i > d_i} \]
\end{description}

\subsection{Aperiodic Scheduling}
\begin{description}
	\item[EDD - Earliest Due Date] Synchronous, independent tasks. Preemption
	does not matter. \\ Complete, minimizes maximum lateness.\\ $\in \BIGO(n
	\log n)$
	\item[EDD without preemption] Optimal iff non-idle
	\item[EDF - Earliest Deadline First] Aperiodic, preemptive.\\ Complete,
	minimizes maximum lateness.\\ $\in \BIGO(n \log n)$, when $k$ tasks arrive,
	additionally $\BIGO(k*n)$
	\item[EDF with Dependent Tasks] Choose earliest deadline among all sources,
	schedule, remove, repeat. \textbf{Not} optimal
	\item[EDF*] Optimal for preemptive, asynchronous scheduling.\\
	Either modify the deadlines of all non-sinks:
	\[ d_i^* = \min(d_i, \min\{d_j^* - C_j : J_i \text{ depends on } J_j\}) \]
	Or modify the release time of all non-sources
	\[ r_j^* = \max(r_j, \max\{r_i^* + C_i : J_i \text{ depends on } J_j\}) \]
	\item[LDF with Dependent Tasks] Choose latest deadline among all sinks,
	schedule, remove, repeat. Optimal. Off-line computable $\in \BIGO(n * \max
	\{|E|, \log n\})$.
	\item[Bratley's algorithm] For each schedulable task: Check whether when
	scheduled another task will miss the deadline. Yes $\rightarrow$ search for
	other schedule. No $\rightarrow$ recursion. \\ 
	Exponential time $\rightarrow$ use off-line.
\end{description}

\subsection{Periodic Scheduling}
\begin{description}
	\item[Processor Utilization] \ 
	\[ U = \sum_{i=1}^n \frac{C_i}{T_i} \]
	\item[Utilization Bound (single processor)] $U < U_{bnd} \rightarrow$
	schedulable.\\
	$U_{bnd} \leq U \leq 1 \rightarrow$ maybe schedulable.\\
	$U > 1 \rightarrow$ not schedulable.
	\item[EDF - Earliest Deadline First] Works, but is too slow for practice.
	$U_{bnd} = 1$ for $D_i \geq T_i$
	\item[RM - Rate Monotonic] Assume $T_i = D_i$. prio$(\tau_i) = T_i^{-1}$.\\
	Optimal among fixed priority schedulers. $U_{bnd} = \ln 2 \approx 0.69$.\\
	Exact check:
	\begin{enumerate}
		\item Sort by priorities, \ie, $i = 1$ has highest priority.
		\item Initialize: $R^0_i = C_i$
		\item Iteration $j$: For each $i > 1$ where $i$ is not a fix-point yet:
			\begin{enumerate}
				\item \ 
				\[ R_i^{j+1} = C_i + \sum_{k=1}^{i-1} C_k \left\lceil \frac{R_i^j}{T_k}
				\right\rceil \]
			\end{enumerate}
		\item Repeat until all $i$s are fix-points.
		\item If any $R_i$ is greater than its $C_i \rightarrow$ not schedulable.
		\item Else: RM-schedulable. 
	\end{enumerate}
	With critical resources see next section.
	\MAX{Include example}
\end{description}

\subsection{Scheduling with Resources}

\begin{description}
	\item[Priority Inheritance] If job $J_1$ needs a resource held by a lower
	priority job $J_2$, then $J_2$ inherits $J_1$'s priority.
	\item[Direct Blocking] Job blocked by missing resource held by lower priority
	job.
	\item[Push-Through Blocking] Job $J_m$ is blocked by lower priority job $J_l$,
	that blocks higher priority job $J_h$ directly. Therefore, $J_m$ has to wait
	for $J_l$ and $J_h$ to finish.
	\item[Schedulability check] Use 
	\[ R_i^{j+1} = C_i + B_i + \sum_{k=1}^{i-1} C_k \left\lceil \frac{R_i^j}
	{T_k} \right\rceil\]
	instead, where $B_i$ is the maximum blocking time.
	\item[Blocking Time Approximation] $D_{j,k}$: Duration of longest critical
	section of task $\tau_k$ using critical section $S_k$.\\
	Jobs ordered by priority, $n$ tasks, $m$ critical sections.
	\[ B_i \leq \sum_{j = i+1}^n \max_k\{ D_{j,k} | C(S_k) \leq P_i \} \]
	\[ B_i \leq \sum_{k = 1}^m \max_{j>i}\{ D_{j,k} | C(S_k) \leq P_i \} \]
	\item[Exact Blocking Time] Exhaustive search, exponential time.
	\item[Priority Ceiling] $C(S) = \max\{\text{prio}(J)|\text{J can lock} S \}$
	\item[Priority Ceiling Protocol] J can lock $S$ if its priority id \emph
	{greater} than the \emph{ceilings} of all semaphores locked by other tasks.\\
	If not the case, $J$ gets priority of this very task.\\
	When exiting the critical section, the priority is set to highest priority of a
	job blocked by the exiting job.
\end{description}

\subsection{Multiprocessor Scheduling with preemption}
\begin{description}
	\item[Schedulability] $U \leq n$
	\item[Game Board Representation] x-Axis: Laxity \\
	y-Axis: Remaining computation time
	\item[Least Laxity First] Optimal.
	\item[Fully Run] $FR(t,k) = \{i \in M | D_i(t) \leq k \}$
	\item[Partially Run] $PR(t,k) = \{i \in M | L_i(t) \leq k \land D_i(t) > k \}$
	\item[Need Not Be Run] $NN(t,k) = \{i \in M | L_i(t) > k \}$
	\item[Surplus Computing Power] \ 
	\[ SCP(t,k) = (k * n) - \sum_{i \in FR(t,k)} C_i(t) - \sum_{i \in PR(t,k)} (k -
	L_i(t)) \]
	$SCP(0,k) \geq 0 \forall k > 0$ is a necessary and sufficient condition on
	schedulability.
	\item[Scheduling idea] See slide 14:36, 14:39
	\item[Operating Systems] 14:43, 14:45, 14:46, 14:47, 14:50
\end{description}