\subsection*{VHDL - Syntax and Stuff}
\begin{description}
	\item[Entity] Comparable to an interface without actual concrete implementation
	\begin{lstlisting}[language=vhdl]
entity full_adder is
port(a,b,carry_in: in Bit;
sum, carry_out: out Bit);
end full_adder;
	\end{lstlisting}

	\item[Architecture] Multiple architectures for one entity, last (\ie, most
	recently analyzed) one is used.
	\item[Behavior]\ 
	\begin{lstlisting}[language=vhdl]
architecture behavior of full_adder is
begin
sum <= (a xor b) xor carry_in after 10 Ns;
carry_out <= (a and b) or (a and carry_in) 
	or (b and carry_in) after 10 Ns;
end behavior;
	\end{lstlisting}

	\item[Structure]\ 
	\begin{lstlisting}[language=vhdl]
architecture structure of full_adder is 
component half_adder
port (in1,in2:in Bit; carry:out Bit; sum:out Bit); 
end component; 

component or_gate
port (in1, in2:in Bit; o:out Bit);
end component;

signal x, y, z: Bit; -- local signals
begin
i1: half_adder port map (a, b, x, y);
i2: half_adder port map (y, carry_in, z, sum); 
i3: or_gate port map (x, z, carry_out);
end structure;
	\end{lstlisting}
\MAX{Add additional information like about clocks, bit vectors and the like.}
	
	\item[Process] cannot have subprocesses. Executed sequentially.
	\begin{lstlisting}[language=vhdl]
process[(sensitivity list)]
	declarations (of local variables, NO signals)
begin
	statements
end process
	\end{lstlisting}

	\item[Wait]\ 
	\begin{description}
		\item[on \textit{signals}] waits until at least one signal changes.
		\item[until \textit{condition}] waits until condition is satisfied
		\item[for \textit{duration}] waits for the specified amount of time.
		\item[\textit{nothing}] waits forever...
	\end{description}
	\item[Signals] assigned with $<=$. Can have multiple assignments with
	delays. 
	\begin{lstlisting}[language=vhdl]
sig <= reject 3ns inertial not a after 10 ns;
sig <= transport not a after 10 ns;
sig <= '0', '1' after 2ns, '0' after 3 ns;
	\end{lstlisting}
	\item[Variables] assigned with $:=$.
	\item[Parameterized Hardware]\ 
	\begin{lstlisting}[language=vhdl]
architecture RTL2 of SHIFT1024 is
component DFF
	port ( RSTn, CLK, D: in std_logic;
	Q : out std_logic ); 
end component;
signal T: std_logic_vector(n-2 downto 0);
begin
g0: for i in n-1 downto 0 generate
	g1: if (i = n-1) generate
		bit1023 : DFF port map (RSTn,CLK,SI,T(n-2)); 
		end generate;
	g2: if (i>0) and (i<n-1) generate
		bitm : DFF port map (RSTn,CLK,T(i),T(i-1));
		end generate;
	g3: if (i=0) generate
		bit0 : DFF port map (RSTn,CLK,T(0),SO); 
		end generate;
	end generate; 
end RTL2;
	\end{lstlisting}
	Instantiate with 
	\begin{lstlisting}[language=vhdl]
component SHIFTn is
generic ( n : positive);
port ( RSTn, CLK, SI : in std_logic;
SO : out std_logic ); 
end component;
...
begin
...
Shift32comp : SHIFTn 
generic map (n => 32) 
port map(RSTn => ... );
end
	\end{lstlisting}
\end{description}

\subsection*{VHDL - Semantics and Stuff}

\begin{description}
	\item[Signals] Assigned concurrently, global scope, has time dimension.
	\item[Delay Models] \ \\
	\textbf{inertial delay:} default; Absorbs pulses of size
	less than the delay time. Using reject x all pulses of length less than x are
	absorbed. \\
	\textbf{transport delay:} Every pulse is translated.
	When inserted in transaction list, all changes for this signal at a later or
	equal point in time are removed
	\MAX{Introduce more detailed execution algorithm}
	\item[Variables] Assigned sequentially, local scope, cannot be in sensitivity
	lists, no time dimension.
	\item[Process] Runs in a loop, only stops at waits or if there is a sensitivity
	list.
	\item[Simulation] \ 
	\begin{enumerate}
		\item Initialization (incl. assign variables)
		\item \label{sim:item:update_time}Update time
		\item Assign signals
		\item Evaluate processes if resumed
		\item Changes pending: goto \ref{sim:item:update_time}
		\item No changes pending: end simulation.
	\end{enumerate}
	\item[Transaction list] Pending assignments. 
	(clk, `1', 10ns). Time is absolute, not relative.
	\item[Process activation list] Pending resumed processes ($p_1$, 10ns).
	\item[Delta Steps] Two steps can be performed after each other at the same
	time.
	\item[Conflict Resolution] If two values are assigned to the same signal at the
	same time, the stronger one wins, \ie, the one further at the beginning of the
	signal's domain specification
\end{description}

\subsection*{Circuit Synthesis}
\begin{description}
	\item[Abstraction Layer Hierarchy] \ 
	\begin{description}
		\item[System] ISA description, block diagrams
		\item[Algorithmic] Functional Implementation: Matlab, C, Cobol
		\item[Behavioral] Circuit description: VHDL, Verilog
		\item[Register-Transfer (RTL)] Registers, logical functions, Netlist (graph)
		\item[Logical Gate] Netlist, 1-bit registers, basic logic gates
		\item[Physical Gate] Netlist of target gates, \eg LUTs 
		\item[Switch] Netlist of individual transistors.
	\end{description}

	\item[Non-synthesizable] delays, file operations, initial values, dynamic data
	structures, unresolved generic values
\end{description}

\subsection*{FPGA}
<advertisement> \\
When compared with ASICs
Faster prototyping; less expensive, even for a low volume of products, timing
very well explored -> important for real time applications \\
</advertisement>\\
Consists of blocks with x-bit lookup tables with $2^x$ outputs and $y$ 1 bit
registers
