\newcommand{\XSTAR}{\ensuremath{x^*} }
\newcommand{\RN}{\ensuremath{\mathds{R}^n}}
\newcommand{\ST}{\mbox{s.t. }}
\newcommand{\WLOG}{\mbox{w.l.o.g }}
\newcommand{\mst}{\ \mbox{s.t. }\ }
\newcommand{\ie}{\mbox{i.e.}}
\newcommand{\eg}{\mbox{e.g. }}
\newcommand{\dftPolyhedron}{\ensuremath{\{x|Ax\geq b\}}}
\newcommand{\REPLACE}{\leadsto}
\newcommand{\IMPLIESNOT}{\centernot\implies}
\newcommand{\CONVEXLAMBDA}{\ensuremath{\lambda \in (0,1)}}
\newcommand{\NORM}[1]{\ensuremath{\Vert #1 \Vert}}
\newcommand{\ABS}[1]{\ensuremath{|#1|}}
\newcommand{\CARDINALITY}[1]{\ensuremath{|#1|}}
\newcommand{\BS}{\textbackslash}
\newcommand{\OLD}[1]{(#1)\textsuperscript{old}}
\newcommand{\NEW}[1]{(#1)\textsuperscript{new}}
\renewcommand{\epsilon}{\varepsilon} % CAUTION!
\DeclareMathOperator{\Span}{span}
\DeclareMathOperator{\SPAN}{span}
\DeclareMathOperator{\ARGMIN}{argmin}
\DeclareMathOperator{\RANK}{rank}
\DeclareMathOperator{\BIGO}{\mathcal{O}}
\DeclareMathOperator{\LEXG}{>_{lex}}
\DeclareMathOperator{\LEXL}{<_{lex}}
\DeclareMathOperator{\SINC}{sinc}
% \renewcommand{\phi}{\varphi} % CAUTION!

\definecolor{CommentColor}{rgb}{0.90,0.16,0} %red
\newcommand{\MAX}[1]{\textbf{\color{CommentColor} /* (max) #1  (max) */}}
\newcommand{\BEN}[1]{\textbf{\color{CommentColor} /* (max) #1  (max) */}}
\newcommand{\MARLENE}[1]{\textbf{\color{CommentColor} /* (max) #1  (max) */}}
\newcommand{\MAXI}[1]{\textbf{\color{CommentColor} /* (max) #1  (max) */}}

\newcommand{\EXERCISE}[1]{\section*{Exercise #1}}
\newcommand{\AUFGABE}[1]{\section*{Aufgabe #1}}

\renewcommand{\quote}[1]{``#1''}

\newcommand{\nl}{\newline}

\newcommand{\NOTE}[1]{\quad \mbox{{\color{lightgray} #1}}}

\newcommand{\LP}[1]{%
	\begin{center}
		\begin{minipage}{0.5\textwidth}
			\begin{flushleft}
				#1
			\end{flushleft}
		\end{minipage}
	\end{center}
}%

% \newenvironment{caseDestinction}[1]{\textbf{#1} \begin{addmargin}[1cm]{0em}}{\end{addmargin}}

\newcommand{\dftLPfst}{
\LP{
$\min c^Tx$ \\
\quad $Ax \geq b$
}
}

\newcommand{\stdLP}{
\LP{
$\min c^Tx$ \\
\quad $Ax = b $\\
\quad $x \geq 0$
}
}

\newcommand{\twopartdef}[3]
{
	\left\{
		\begin{array}{ll}
			#1 & \mbox{if } #2 \\
			#3 & \mbox{otherwise} 
		\end{array}
	\right.
}

\newcommand{\threepartdef}[5]
{
	\left\{
		\begin{array}{lll}
			#1 & \mbox{if } #2 \\
			#3 & \mbox{if } #4 \\
			#5 & \mbox{otherwise}
		\end{array}
	\right.
}