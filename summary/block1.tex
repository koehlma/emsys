
\subsection{State Representation}

\begin{description}
	\item[State Description] See slide 2:11
	\item[Execution Existence] The defining functions are continuous.
	\item[Execution Uniqueness] Lipschitz-continuous, i.e., $\NORM{f(u) - f(v)} 
	\leq K \NORM{u - v}$ for some constant $K$.
	\item[Equilibrium] All derivatives are 0 for a given state $x$.
	\item[Lyapunov Stability] For each initial deviation there is a constant 
	bound for all state variables for any future point in time. (Assumption: No
	inputs)
	\item[Lyapunov Asymptotic Stability] For each deviation within a certain bound
	the state will converge to the equilibrium again.
	\item[Bounded-Input-Bounded-Output Stability] For every bounded input, the output is bounded
	\item[Finite State Machine] See slide 3:18
	\item[Mealy] Output depends on input and state.
	\item[Moore] Output depends on state only.
	\item[Extended State Machine] Finite state machine with variables. Note: A
	state is the state coupled with its variables' evaluations. See slide 3:20
	\item[AND Superstate] All states are active at once.
	\item[OR Superstate] Only one state is active. History stores from where state
	has been left.
	\item[Hybrid Automaton] Discrete steps (\eg update state) and continuous steps
	(\eg updating distance, time etc.).
	\item[Super Step Semantics] Apply discrete steps until fixed point, then
	continuous steps until discrete step applicable.
	\item[Zeno] Execution is Zeno if sum over all steps' durations in bounded by a
	constant. State is Zeno if all executions staring in this state are Zeno.
\end{description}
