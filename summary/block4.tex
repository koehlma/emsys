\subsection*{Communication}

\begin{description}
    \item[Scalability] Limitlessly extensible, analysis independent of the
    concrete number of nodes.
    \item[Composability] Modular design, $P$ holds for all subsystems $\implies$ 
    P holds for the whole system.
    \item[Fault Tolerance] Achieved by ...
        \begin{itemize}
            \item defining error-containment regions, \ie, multiple outputs,
            some might be wrong.
            \item masking faults, \ie, a component might fail, but the output is
            still correct.
            \item replication of components, \ie, static redundancy to obtain
            the correct result even though some components fail.
            \item fault detection and recovery, \ie, re-run computation if
            result is wrong.
        \end{itemize}
    \item[Safe] sufficiently low probability of causing serious harm. E.g.\ stop
    service in case of an error.
    \item[Reliable] sufficiently high probability of delivering intended
    service. Cannot simply stop service.
    \item[Fault] Primary cause of failure.\\
    Example: landing gear does not report that gear is down.
    \item[Error] Wrong internal state of a (sub-)system. \\
    Example: thrust reverters blocked by control software.
    \item[Failure] Deviation from intended service. \\
    Example: Braking distance increased drastically, millions die.
    \MAX{Include slide 9:25, 9:27, 9:29 here...}
    \item[Static: $n$-Modular Redundancy (TMR)] $n$ identical modules, one
    majority-voter.
    High cost, size, weight, energy. Protects against systematic failure.
    \item[Multi Stage TMR] Multiple modules, multiple voters, their output is
    again input for multiple modules.
    \item[Dynamic: Standby Spare Arrangement] Two modules, one fault detector
    operating a switch.
    \item[Hot/Cold Standby] \textbf{Hot:} Fast switching, increased power
    consumption. \textbf{Cold:} Reduced power consumption and wear out. Slow
    switching.
    \item[Hybrid Redundancy] see slide 9:36 \MAX{TODO}
    \item[Failure Models] see slide 9:39 \MAX{TODO}
    \item[Byzantine Generals] see slides 9:40-48
    \item[Jitter] Variation in latency
    \item[Latency] Distribution/Communication overhead
    \item[Explicit Flow Control] 
    \begin{itemize}
        \item Receiver controls pace via acknowledgments,
        start/stop signals, maskable interrupts, or real-time observable receiver
        state (e.g. bus snooping).
        \item Receiver might have to retry communication.
        \item Prone to thrashing.
    \end{itemize}
    \item[Implicit Flow Control] 
    \begin{itemize}
        \item Pace determined at design time or startup based on \eg pings.
        \item No retrying, thrashing
    \end{itemize}
    \item[Bus: Master-Slave Arbitration]\ \\
    \textbf{+}: Simple, bounded latency, easily adaptable.\\
    \textbf{-}: Requires fault tolerance, polling consumes bandwidth, number of
    nodes fixed or requires bandwidth to detect.
    \item[Bus: Fault Detection] Faulty/slow slave: master issues timeout based
    on worst time response time. Faulty master: Heartbeats allow detection.
    Dynamic redundancies.
    \item[Time Division Multiplexed Access (TDMA)] See slide 11:33 for 
    (dis-)advantages \MAX{TODO}
    \begin{itemize}
        \item No collisions.
        \item Global time frame; clock sync necessary.
        \item Transmit in dedicated time slot.
    \end{itemize}
    \item[Carrier Sense Multiple Access (CSMA)]
    \begin{itemize}
        \item Transmit when bus is idle.
        \item Collision detection $\rightarrow$ back off dedicated time. See
        slide 11:36 for (dis-)advantages.
        \item Collision Avoidance: Bus arbitration mechanism avoids collisions.
    \end{itemize}
    \item[Controller Area Network (CAN)] see slide 11:39 \MAX{todo}
    \item[I2C] see slide 11:40-43 \MAX{todo}
    \item[FlexRay] Each cycle: static and dynamic segment to combine advantages.
    See slide 11:56
\end{description}

\subsection*{Measuring Reliability}

Let $(*)$ be the assumption of having an exponential distribution, \ie, a
constant failure rate.
\begin{description}
    \item[Limitations] The following always assumes independence of failures,
    which is often not the case. State based model more realistic.
    \item[Density Function for time until first failure] constant failure rate
    $\lambda:$ $f(t) = \lambda e^{-\lambda t}$.
    \item[Probability of being faulty] $F(t) = \Pr(T \leq t) = \int^t_0f(\tau)
    d\tau = 1 - e^{-\lambda t}$
    \item[Reliability] not yet faulty. $R(t) = \int_0^\infty f(\tau) d\tau = 1 -
    F(t) \overset{(*)}{=} e^{-\lambda t}$
    \item[Failure rate $\lambda$] $\frac{f(t)}{R(t)}$
    \item[Mean Time To Failure (MTTF)] = $\int_0^\infty t f(t) dt \overset{(*)}
    {=} \frac{1}{\lambda}$
    \item[Mean Time To Repair (MTTR)] Time to repair a fault that has occurred.
    \item[Mean Time Between Failures (MTBF)] = MTTF + MTTR
    \item[Availability] $A = \frac{\text{MTTF}}{\text{MTBF}}$
    \item[FIT] expected number of failures in $10^9$ device hours.
    Failure in time, \textbf{NOT} failures in terra.
    \item[Failure Mode and Effect Analysis (FMEA)] Start with base components,
    analyze hierarchically upwards.
    \item[Serial Composition] $R(t) = \Pi^N_{i=1}R_i(t)$
    \item[Parallel Composition] $R(t) = 1 - \Pi^N_{i=1}(1-R_i(t))$
    \item[Minimal Cuts] Let C be the set of minimal cuts.
    \[ R(t) \geq 1 - \sum_{S \in C} \Pi_{i \in S}(1 - R_i(t)) \]
    \item[Minimal Tie Sets] Let T be the set of minimal ties, \ie, the set of
    sets of components, whose simultaneous correctness guarantees overall
    correctness.
    \[ R(t) \leq \sum_{S \in T} \Pi_{i \in S} R_i(t) \]
    \item[Fault Tree Analysis (FTA)] Transform graph into tree: OR-node if single error results in
        overall failure. AND-node if failure of all children required to get an
        overall failure.
    \item[Mocus Algorithm] Uses transformed tree to compute minimal cut sets.
    \begin{enumerate}
        \item Initialize $1 \times 1$ Matrix with root operator.
        \item \label{alg:mocus:rec} Replace operator in matrix. \\
        If AND $\rightarrow$ duplicate row. Replace operator by children.\\
        If OR  $\rightarrow$ duplicate col. Replace operator by children.
        \item If there is an operator left, goto \ref{alg:mocus:rec}\\
        \item Each \emph{column} is a cut set. Remove duplicates.
    \end{enumerate}
    \item[Markov Chain] Automation with initial probabilities $i(q)$ and
    transition probabilities $t(q, q')$ (0 if not connected). Chance to be in a
    state $q$:
    \[ s_n(q) = \twopartdef{i(q)}{n = 0}{\sum_{r \in Q}(s_{n-1}(r)t(r,q))} \]
\end{description}

\subsection*{Event vs. Time Triggered Communication}
\begin{tabular}{l |c | c}
                    & Event Triggered (ET)  & Time Triggered (TT) \\
\hline
Interrupt based     & yes                   & no \\
Shares ...          & events                & data\\
Order significant   & yes                   & no\\
processed only once & yes                   & not nec.\\
Collisions possible & yes                   & no (avoided at design level)\\
Overloading possible& yes                   & no
\end{tabular}

